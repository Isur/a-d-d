\documentclass[12pt,a4paper]{article}
\usepackage[polish]{babel}
\usepackage[T1]{fontenc}
\usepackage{lmodern}
\usepackage[utf8x]{inputenc}
\usepackage{hyperref}
\usepackage{url}
\usepackage{graphicx}
\usepackage{listings}
%\usepackage{xcolor}
\usepackage{color}
\usepackage{float}
\usepackage{multicol}
\usepackage{tikz}
\usepackage{makecell}
\renewcommand{\arraystretch}{1.5}
\title{Academic Data Deliverer\\Programowanie Obiektowe i Graficzne}
\author{Artur Bednarczyk, Dawid Grajewski, Damian Kwaśniok\\Politechnika Śląska\\Wydział Matematyki Stosowanej\\Informatyka, semestr IV}
\date{\today}

\begin{document}
	\maketitle
	\begin{figure}[H]
		\centering
		\includegraphics[width=0.5\linewidth]{LOGO2}
		\label{fig:logo}
	\end{figure}
	\clearpage
	\tableofcontents
	\clearpage
	\section{Harmonogram prac}
	\begin{tabular}{|p{0.7\textwidth}|c|c|}
	\hline
	Co & Kto & Kiedy \\	\hline\hline
	Połączenie z bazą danych & Grajewski & 10.05.2018 \\ \hline
	Szkielet struktury MVP & Bednarczyk & 23.05.2018 \\	\hline
	Wyśrodkowanie elementów widocznych z możliwością rozszerzenia okna & Bednarczyk & 23.05.2018 \\ \hline
	Logowanie & Grajewski & 26.05.2018 \\ \hline\hline
	Dokumentacja & \makecell{Bednarczyk \\Grajewski\\ Kwaśniok} & Cały czas\\ \hline
	\end{tabular}
	\clearpage
	\section{Opis projektu}
		\subsection{Opis}
			Aplikacja dla studentów umożliwiająca łatwą i szybką wymianę notatek.
		\subsection{Funkcjonalności}
			\subsubsection{Logowanie i Rejestracja}
				Zaloguj się gdziekolwiek jesteś.
			\subsubsection{Przypasanie do grup}
				Zapisz się do grupy Twojej uczelni/wydziału/kierunku.
			\subsubsection{Notatki}
				Oglądaj i dodawaj nowe notatki.
	\section{Rozwiązania}
		\subsubsection{Technologie}
			.NET, C\#, MySQL, Internet.
		\subsubsection{Oprogramowanie}
			Visual Studio 2015, Heroku.
	\section{Projekt UI}
		\subsection{Logowanie}
			Obrazek.png
		\subsection{Rejestracja}
			Obrazek.png
		\subsection{Profil}
			Obrazek.png
		\subsection{Notatka}
			Obrazek.png
	\section{Implementacja}
		\subsection{Podział projektu na pliki}
			Folder : pliki itd.
		\subsection{Schemat Modelu Obiektowego}
			Może być UML		
		\subsection{Interfejsy}
			Opisy wszystkich interfejsów
		\subsection{Algorytm/Rozwiązanie}
			Jakiś ciekawy, własny pomysł na coś, jakiś algorytm, coś fajnego.
	\section{Testy}
\end{document}